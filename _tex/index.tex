% Options for packages loaded elsewhere
\PassOptionsToPackage{unicode}{hyperref}
\PassOptionsToPackage{hyphens}{url}
\PassOptionsToPackage{dvipsnames,svgnames,x11names}{xcolor}
%
\documentclass[
]{agujournal2019}

\usepackage{amsmath,amssymb}
\usepackage{iftex}
\ifPDFTeX
  \usepackage[T1]{fontenc}
  \usepackage[utf8]{inputenc}
  \usepackage{textcomp} % provide euro and other symbols
\else % if luatex or xetex
  \usepackage{unicode-math}
  \defaultfontfeatures{Scale=MatchLowercase}
  \defaultfontfeatures[\rmfamily]{Ligatures=TeX,Scale=1}
\fi
\usepackage{lmodern}
\ifPDFTeX\else  
    % xetex/luatex font selection
\fi
% Use upquote if available, for straight quotes in verbatim environments
\IfFileExists{upquote.sty}{\usepackage{upquote}}{}
\IfFileExists{microtype.sty}{% use microtype if available
  \usepackage[]{microtype}
  \UseMicrotypeSet[protrusion]{basicmath} % disable protrusion for tt fonts
}{}
\makeatletter
\@ifundefined{KOMAClassName}{% if non-KOMA class
  \IfFileExists{parskip.sty}{%
    \usepackage{parskip}
  }{% else
    \setlength{\parindent}{0pt}
    \setlength{\parskip}{6pt plus 2pt minus 1pt}}
}{% if KOMA class
  \KOMAoptions{parskip=half}}
\makeatother
\usepackage{xcolor}
\setlength{\emergencystretch}{3em} % prevent overfull lines
\setcounter{secnumdepth}{5}
% Make \paragraph and \subparagraph free-standing
\makeatletter
\ifx\paragraph\undefined\else
  \let\oldparagraph\paragraph
  \renewcommand{\paragraph}{
    \@ifstar
      \xxxParagraphStar
      \xxxParagraphNoStar
  }
  \newcommand{\xxxParagraphStar}[1]{\oldparagraph*{#1}\mbox{}}
  \newcommand{\xxxParagraphNoStar}[1]{\oldparagraph{#1}\mbox{}}
\fi
\ifx\subparagraph\undefined\else
  \let\oldsubparagraph\subparagraph
  \renewcommand{\subparagraph}{
    \@ifstar
      \xxxSubParagraphStar
      \xxxSubParagraphNoStar
  }
  \newcommand{\xxxSubParagraphStar}[1]{\oldsubparagraph*{#1}\mbox{}}
  \newcommand{\xxxSubParagraphNoStar}[1]{\oldsubparagraph{#1}\mbox{}}
\fi
\makeatother


\providecommand{\tightlist}{%
  \setlength{\itemsep}{0pt}\setlength{\parskip}{0pt}}\usepackage{longtable,booktabs,array}
\usepackage{calc} % for calculating minipage widths
% Correct order of tables after \paragraph or \subparagraph
\usepackage{etoolbox}
\makeatletter
\patchcmd\longtable{\par}{\if@noskipsec\mbox{}\fi\par}{}{}
\makeatother
% Allow footnotes in longtable head/foot
\IfFileExists{footnotehyper.sty}{\usepackage{footnotehyper}}{\usepackage{footnote}}
\makesavenoteenv{longtable}
\usepackage{graphicx}
\makeatletter
\newsavebox\pandoc@box
\newcommand*\pandocbounded[1]{% scales image to fit in text height/width
  \sbox\pandoc@box{#1}%
  \Gscale@div\@tempa{\textheight}{\dimexpr\ht\pandoc@box+\dp\pandoc@box\relax}%
  \Gscale@div\@tempb{\linewidth}{\wd\pandoc@box}%
  \ifdim\@tempb\p@<\@tempa\p@\let\@tempa\@tempb\fi% select the smaller of both
  \ifdim\@tempa\p@<\p@\scalebox{\@tempa}{\usebox\pandoc@box}%
  \else\usebox{\pandoc@box}%
  \fi%
}
% Set default figure placement to htbp
\def\fps@figure{htbp}
\makeatother
% definitions for citeproc citations
\NewDocumentCommand\citeproctext{}{}
\NewDocumentCommand\citeproc{mm}{%
  \begingroup\def\citeproctext{#2}\cite{#1}\endgroup}
\makeatletter
 % allow citations to break across lines
 \let\@cite@ofmt\@firstofone
 % avoid brackets around text for \cite:
 \def\@biblabel#1{}
 \def\@cite#1#2{{#1\if@tempswa , #2\fi}}
\makeatother
\newlength{\cslhangindent}
\setlength{\cslhangindent}{1.5em}
\newlength{\csllabelwidth}
\setlength{\csllabelwidth}{3em}
\newenvironment{CSLReferences}[2] % #1 hanging-indent, #2 entry-spacing
 {\begin{list}{}{%
  \setlength{\itemindent}{0pt}
  \setlength{\leftmargin}{0pt}
  \setlength{\parsep}{0pt}
  % turn on hanging indent if param 1 is 1
  \ifodd #1
   \setlength{\leftmargin}{\cslhangindent}
   \setlength{\itemindent}{-1\cslhangindent}
  \fi
  % set entry spacing
  \setlength{\itemsep}{#2\baselineskip}}}
 {\end{list}}
\usepackage{calc}
\newcommand{\CSLBlock}[1]{\hfill\break\parbox[t]{\linewidth}{\strut\ignorespaces#1\strut}}
\newcommand{\CSLLeftMargin}[1]{\parbox[t]{\csllabelwidth}{\strut#1\strut}}
\newcommand{\CSLRightInline}[1]{\parbox[t]{\linewidth - \csllabelwidth}{\strut#1\strut}}
\newcommand{\CSLIndent}[1]{\hspace{\cslhangindent}#1}

\usepackage{url} %this package should fix any errors with URLs in refs.
\usepackage{lineno}
\usepackage[inline]{trackchanges} %for better track changes. finalnew option will compile document with changes incorporated.
\usepackage{soul}
\linenumbers
\makeatletter
\@ifpackageloaded{float}{}{\usepackage{float}}
\floatstyle{plain}
\@ifundefined{c@chapter}{\newfloat{suppfig}{h}{losuppfig}}{\newfloat{suppfig}{h}{losuppfig}[chapter]}
\floatname{suppfig}{Figure S}
\newcommand*\quartosuppfigref[1]{Figure \hyperref[#1]{S\ref{#1}}}
\@ifpackageloaded{caption}{}{\usepackage{caption}}
\DeclareCaptionLabelFormat{quartosuppfigreflabelformat}{#1#2}
\captionsetup[suppfig]{labelformat=quartosuppfigreflabelformat}
\newcommand*\listofsuppfigs{\listof{suppfig}{List of Supplementary Figures}}
\makeatother
\makeatletter
\@ifpackageloaded{caption}{}{\usepackage{caption}}
\AtBeginDocument{%
\ifdefined\contentsname
  \renewcommand*\contentsname{Table of contents}
\else
  \newcommand\contentsname{Table of contents}
\fi
\ifdefined\listfigurename
  \renewcommand*\listfigurename{List of Figures}
\else
  \newcommand\listfigurename{List of Figures}
\fi
\ifdefined\listtablename
  \renewcommand*\listtablename{List of Tables}
\else
  \newcommand\listtablename{List of Tables}
\fi
\ifdefined\figurename
  \renewcommand*\figurename{Figure}
\else
  \newcommand\figurename{Figure}
\fi
\ifdefined\tablename
  \renewcommand*\tablename{Table}
\else
  \newcommand\tablename{Table}
\fi
}
\@ifpackageloaded{float}{}{\usepackage{float}}
\floatstyle{ruled}
\@ifundefined{c@chapter}{\newfloat{codelisting}{h}{lop}}{\newfloat{codelisting}{h}{lop}[chapter]}
\floatname{codelisting}{Listing}
\newcommand*\listoflistings{\listof{codelisting}{List of Listings}}
\makeatother
\makeatletter
\makeatother
\makeatletter
\@ifpackageloaded{caption}{}{\usepackage{caption}}
\@ifpackageloaded{subcaption}{}{\usepackage{subcaption}}
\makeatother

\usepackage{bookmark}

\IfFileExists{xurl.sty}{\usepackage{xurl}}{} % add URL line breaks if available
\urlstyle{same} % disable monospaced font for URLs
\hypersetup{
  pdftitle={Effects of Riparian Grazing on Distinct Phosphorus Sources},
  pdfauthor={Alexander J Koiter; Tamaragh Y Malone},
  pdfkeywords={Phosphorus, Grazing, Riparian},
  colorlinks=true,
  linkcolor={blue},
  filecolor={Maroon},
  citecolor={Blue},
  urlcolor={Blue},
  pdfcreator={LaTeX via pandoc}}


\journalname{TBD}

\draftfalse

\begin{document}
\title{Effects of Riparian Grazing on Distinct Phosphorus Sources}

\authors{Alexander J Koiter\affil{1}, Tamaragh Y Malone\affil{2}}
\affiliation{1}{Brandon University, Department of Geography and
Environment, Brandon, MB, }\affiliation{2}{Brandon University,
Department of Biology, Brandon, MB, }
\correspondingauthor{Alexander J Koiter}{koitera@brandonu.ca}


\begin{abstract}
Riparian areas play an important role in maintaining water quality in
agricultural watersheds by buffering sediment, nutrients, and other
pollutants. Recent studies have shown that riparian areas are less
effective as buffers and, in some cases, are a net source of phosphorus
(P) in cold climates. This study assessed the impact of cattle grazing
or harvesting of riparian areas on the spatial and vertical distribution
of P. This study measured the water-extractable phosphorus (WEP) in four
distinctive sources: biomass, litter, organic layer, and Ah horizon in
three riparian locations extending from the edge of the waterbody to the
the field edge. Four treatments were examined: 1) control; 2) grazing;
3) high density grazing; and 4) mowing. Prior to implementing the
treatments, the Ah (0-10cm) soil was the largest pool of WEP (42.5 mg
m\textsuperscript{-2}, \textasciitilde44\%); however, the biomass (i.e.,
standing vegetation) was a considerable proportion of the total (26.3 mg
m\textsuperscript{-2}, \textasciitilde25\%) WEP pool. The litter and
organic layer had median P amounts of 11.1 and 17.7 mg
m\textsuperscript{-2}, respectively. Findings revealed significant
reductions in biomass WEP with median reductions of 10.4 and 18.7 mg
m\textsuperscript{-2} for high-density grazing and mowing treatments,
respectively. This reduction was more pronounced in the lower riparian
locations where there was more biomass available to be grazed or mowed.
There were no detectable changes in the other sources of WEP across all
the treatments. Assessment of the control plots (pre- and
post-treatment) clearly indicate that there is considerable small-scale
spatial variability in P measurements in riparian areas. Overall, the
results of this study suggests that management practices that target
vegetation, including harvesting and autumn short-term grazing, may be
mechanisms to reduce the potential P loss during the snowmelt period.
Studies investigating other important riparian processes that also have
a demonstrated impact on the P mobility including freeze-thaw cycles and
flooding are needed to fully assess the risk of P loss.
\end{abstract}

\section*{Plain Language Summary}
Riparian areas are important for keeping water clean in agricultural
watersheds because they help filter out sediment, nutrients, and other
pollutants. Some recent studies found that in cold climates, like the
Canadian Prairies, riparian areas are not as effective at filtering out
nutrients. Because of the freeze and thaw of soil and vegetation during
the spring snowmelt riparian areas can be a source of phosphorus to the
water instead of removing it. To see if we can reduce the loss of
phosphorus, we looked at different sources of phosphorus in riparian
areas including plants, dead vegetation, and soil. Cattle grazing and
mowing were tested as ways of managing the riparian areas. Both cattle
grazing and mowing reduced the amount of plant-based phosphorus without
increasing the other sources. This shows that letting cows graze in the
fall might be a good way to use this forage and also prevent too much
phosphorus from getting into the water when the snow melts in the
spring.




\textbf{Core ideas}

\begin{itemize}
\tightlist
\item
  Biomass and litter are substantial sources of WEP in riparian areas
\item
  Autumn cattle grazing and mowing treatments reduced the amount of WEP
  in riparian biomass
\item
  No Measurable change in the amount of WEP in the litter, organic
  layer, or Ah horizon post grazing
\item
  Large spatial variability in WEP exists in riparian areas
\end{itemize}

\textbf{Abbreviations}

FTC, freeze-thaw cycle; MBFI, Manitoba Beef and Forage Initiatives; P,
phosphorus; WEP, water extractable phosphorus

\section{Introduction}\label{introduction}

The increasing frequency and extent of algal blooms are typically linked
to increased nutrient loading into lake and rivers. Phosphorus (P)
loading is particularly concerning as this is generally the limiting
nutrient in fresh water systems (Schindler et al., 2012). There have
been many lab and field studies demonstrating the role and functionality
of riparian areas in reducing P loading to surface water in agricultural
settings (Yu et al., 2019). Infiltration, absorption, biological uptake,
microbial activity, and sedimentation are the key processes that
intercept and buffer the delivery of P (Lacas et al., 2005; Owens et
al., 2007; McGuire and McDonnell, 2010). Convergence within the
landscape coupled with climatic/weather conditions creates variability
in hydrologic conditions and pathways, reducing the buffering capacity
of riparian areas and ultimately resulting in reduced, inconsistent,
and/or unsustainable reductions in P loading relative to many controlled
experimental studies (Roberts et al., 2012; Habibiandehkordi et al.,
2017).

In cold climates, the reduced infiltration due to frozen ground, limited
vegetation uptake, and low microbial activity coupled with a flashy
hydrograph during snowmelt creates conditions that further compromise
the buffering capacity of riparian areas (Kieta et al., 2018; Nsenga
Kumwimba et al., 2023). Additionally, research increasingly shows that
riparian areas can contribute P (i.e., net source) to the surrounding
environment (Roberts et al., 2012). The sources of this riparian-derived
P are soil and vegetation. As soil P content increases, so does the risk
of P loss through leaching and runoff (Habibiandehkordi et al., 2019).
Soil P release can be intensified during periods of inundation that
often occur dufing the spring snow melt, due to both to a longer period
of soil-water contact and an increase solubility of iron-bound P as soil
redox conditions lower (i.e., become anaerobic) (Carlyle and Hill, 2001;
Young and Briggs, 2008). Vegetation P can become more mobile through the
mineralization of P from decaying vegetation near the soil surface.
There is also evidence that the longer vegetation-water contact during
periods of inundation will also increase the amount of P leached out of
the the dead vegetation contribute to the P available to be lost during
runoff (Lozier and Macrae, 2017; Liu et al., 2019b). Both the soil and
vegetation P sources can also be affected by freeze-thaw cycles (FTC).
Repeated FTCs result in the cell disruption of microbial and plant
biomass, releasing inter-cellular P to the surrounding environment
(Kieta and Owens, 2019).

Management of riparian areas to maintain or enhance the buffering
capacity of P is typically needed in the long term. Unlike nitrogen (N)
where N can be significantly lost to the atmosphere through the
processes of nitrification and denitrification to offset the continued
input (Lyu et al., 2021), P is only generally lost through runoff or
leaching. Harvesting and removing of biomass from the riparian area can
be a practice to remove P and use the biomass for forage. However,
mechanized biomass harvesting may be impractical or unsafe due to steep
gradients, wet soil, and other obstacles like trees. Livestock grazing
in riparian areas (riparian pastures) is common in the Canadian Prairies
due to the abundance of forage, particularly during drought. Livestock
exclusion from riparian areas has been suggested as a best management
practices to reduce the direct inputs of P, limit bank erosion, and
avoid soil compaction (Krall and Roni, 2023). However, stategies
including alternative water sources, rotational grazing,
timed-controlled grazing, rest-rotation grazing, and corridor fencing
can all reduce those risks (Fitch et al., 2003).

From a surface water quality perspective, understanding the near-surface
P distribution, both vertically and longitudinally, will help develop
and identify best management practices for reducing P loading from
riparian areas. Vertically, there are often four distinctive and
identifiable sources of near-surface P: 1) biomass consisting of living
standing vegetation; 2) litter consisting of fresh (\textasciitilde1-3
yrs) residues; 3) partially to well decomposed organic material; and 4)
mineral soil (Reid et al., 2018). Longitudinally there often is a strong
soil moisture gradient extending from the edge of the waterbody to the
field edge. This results in changes biomass and litter including amount
and composition as well as soil properties including organic matter
content and horizon thickness. A better understanding of the spatial
variability and relative contributions of the different sources of P is
needed to assess the risks and benefits of different management
strategies.

Given the timing and processes of P dynamics within riparian areas in
cold climates, like the Canadian Prairies, reducing the near-surface
concentration of soluble P prior to spring snowmelt would be a strategy
to limit the contribution of P from the riparian area to surface water.
Therefore, the overall aim of this study is to assess the impacts of
short-term autumn cattle grazing and mowing on the sources and
distribution of P in riparian areas. The objectives of this study were
to quantify 1) the vertical profile of WEP using four distinctive P
sources: biomass, litter, organic layer, and Ah horizon; 2) each of the
four distinctive P sources in three riparian locations, near the edge of
the waterbody (lower), close to the field edge (upper), and in between
(middle); and 3) the net change in each of the four sources of WEP in
each riparian location in response to grazing, high density grazing, and
mowing (harvesting) of biomass. Understanding how riparian management
practices affect the different sources of P can be used to help tailor
management strategies in cold climates and ultimately reduce P loss and
improve downstream water quality.

\section{Methods}\label{methods}

\subsection{Site description}\label{site-description}

\textsubscript{Source:
\href{https://alex-koiter.github.io/riparian-grazing-manuscript/index.qmd.html}{Article
Notebook}}

A randomized complete block experimental design was used to assess the
sources of riparian P and investigate how it changes following cattle
grazing or mowing treatments. The four treatments included a control,
graze, high density graze, and mowing. Each treatment was replicated in
riparian areas surrounding four separate Prairie potholes (wetlands).
Samples of each unique source of P, biomass, litter, organic layer, and
Ah horizon, were collected in three locations (upper, mid and lower)
pre- and post-treatment. All samples were analyzed for WEP and the net
change in each of the four distinctive sources of P was evaluated. The
study was replicated across three sequential years using the same plots.

The study was conducted at the Manitoba Beef and Forage Initiatives
(MBFI) research farm (50.06\(^\circ\)N, 99.92\(^\circ\)W; 502 AMSL),
approximately 25 km north of Brandon, Manitoba, Canada, in the Prairie
Pothole region of North America (Figure~\ref{fig-mapr}). The normal
(1981 -- 2010) average daily air temperature was 2.2 \(^\circ\)C, and
the cumulative annual precipitation at Brandon was 474.2 mm, with 24.8
\% falling as snow (Environment and Climate Change Canada, 2024). The
Köppen-Geiger climate classification is cold, without dry season, and
with warm summer (Dfb) (Beck et al., 2018). The region is predominately
has agricultural land use, including annual crops (grains and oil seeds)
and grazing/forage. MBFI is a 260 hectare (ha) research and
demonstration farm with a mix of pasture, hay, and forage/silage
cropland. Prior to the establishment of MBFI the site was part of the
Manitoba Zero Tillage Research Association farm (1993-2014) where annual
crops, including oil seeds and grains, were grown. There are also
numerous small permanent and ephemeral wetlands (potholes) and
associated riparian areas which account for \textasciitilde35\% of the
total farm land (Manitoba Beef \& Forage Initiatives, 2024). The
riparian areas surrounding the larger permanent wetlands are fenced off
to exclude livestock and are not actively managed. Approximately half
the farm has an irregular undulating to hummocky relief (2-5\%) with the
reminder being nearly level (0-2\%). The soils have developed on fine
loamy, moderately calcareous glacial till. The drainage class in upper
slope positions are well to rapidly draining while lower slope and
riparian soils are poorly drained and primarily consist of Humic and
Luvic Gleysols. The surface texture class of the riparian soil is a clay
loam and pH values range from 7.1 to 8.3 with a mean of 7.6. Generally
the surface soil profile can be described by a 1-10 cm organic layer
overlying a 10-18 cm Ah horizon (Podolsky and Schindler, 1993). The
vegetation in the riparian was assessed using the the foliar cover
method for each plot within each of the four riparian areas. There was
considerable variability between riparian areas, plots, and sampling
locations (upper, mid and lower). The four most dominant species
identified were Sow Thistle (\emph{Sonchus arvensis}), Smooth Aster
(\emph{Aster laevis}), Kentucky bluegrass (\emph{Poa pratensis}), and
Smooth Brome (\emph{Bromus inermis}) and the complete assessment can be
found in \quartosuppfigref{suppfig-plant-plot}. All riparian areas
investigated in this study were adjacent to actively grazed pastures.

\begin{figure}[H]

\centering{

\pandocbounded{\includegraphics[keepaspectratio]{index_files/figure-latex/notebooks-05_Map-fig-mapr-output-2.png}}

}

\caption{\label{fig-mapr}Showing a) the location of the study site in
southern Manitoba with an inset map of Canada; and b) the locations of
the four riparian areas included in this study}

\end{figure}%

\textsubscript{Source:
\href{https://alex-koiter.github.io/riparian-grazing-manuscript/notebooks/05_Map-preview.html\#cell-fig-mapr}{Map
of study area}}

\subsection{Experimental design}\label{experimental-design}

Four riparian areas surrounding permanent wetlands were selected
(Figure~\ref{fig-mapr}) and were subdivided into four \textasciitilde450
\(m^2\) plots. Within each riparian area, each plot was randomly
assigned a treatment. The treatments consisted of 1) control, 2) graze,
3) high density graze, and 4) mow and harvest. The grazing treatments
consisted of a five-hour grazing period, with the grazing treatment
having \textasciitilde3.1-3.5 animal units (AU) per plot and the
high-density grazing with \textasciitilde11.75-12 AU. For the mowing
treatment, the vegetation was cut to a height of \textasciitilde10cm,
and the vegetation was manually raked out of the plot. The cattle were
rotated daily over four consecutive days among the four riparian areas
and the grazed plots were fenced on all four sides, including the edge
of the waterbody, and provided with supplemental water. Treatments were
applied early to mid September, before the first frost, in three
consecutive years (2019-2021) (\quartosuppfigref{suppfig-weather-plot})
Within each plot three distinctive sampling locations, or landscape
positions, were established, adjacent to the edge of the waterbody
(Lower), adjacent to the field/pasture (Upper), and the mid-point (Mid).
Samples were collected in each plot and sampling location 1-3 days prior
and immediately adjacent 1-3 days following the treatments (including
the control) to assess the impact of grazing and mowing.

\subsection{Sampling and analysis}\label{sampling-and-analysis}

Four types of samples were collected: 1) biomass, 2) litter, 3) organic
layer, and 4) Ah horizon. Using a 0.25 \(m^2\) quadrate, biomass was
collected by cutting the standing live vegetation and litter by raking
the surface and picking up the previous years growth. Both the biomass
and litter were dried at 40 \(^\circ C\), weighed, and homogenized using
a blade grinder (\textless1cm). A composite of five soil samples was
collected within the same quadrat as the biomass/litter using a 19 mm
diameter soil probe and was divided into the organic layer and the top
10 cm of the Ah horizon. The organic layer and Ah soil were air-dried,
disaggregated with a mortar and pestle, and passed through a 2-mm sieve.
Additional bulk density samples of both the organic layer and Ah and the
depth of the organic layer were collected in 2023. Daily air temperature
and rainfall data were collected from an onsite station
(\quartosuppfigref{suppfig-weather-plot}) (Manitoba Agriculture, 2023).

Water Extractable Phosphorus (WEP), an environmental soil and vegetation
P test, was used to mimic soil P release to runoff water. Dried and
homogenized samples were extracted by shaking (150 RPM) with deionized
water for one hour at a mass to volume ratio of 1:30 for the biomass and
litter samples (1 g) and 1:15 for the organic and Ah samples (2 g).
Extractions were gravity filtered through a Whatman 42 filter followed
by syringe filtration with a 0.45 \(\mu m\) nylon filter. WEP in the
extract was measured spectrophotometrically by the colorimetric
molybdate--ascorbic acid method (Murphy and Riley, 1962; Sharpley et
al., 2006).

The concentration of WEP in the biomass and litter combined with the
mass of material collected from the quadrat was used to calculate the
total WEP (\(mg~kg^{-1}\)). Only the change in concentration
(\(mg~kg^{-1}\)) was measured for the organic layer and Ah horizon. The
vertical profile of WEP within the riparian area was assessed using
samples collected before treatments were implemented across the 3-year
study. The total WEP in the organic layer and Ah were estimated using
the bulk density and depth measurements collected in 2023
(Figure~\ref{fig-vertical-wep} b).

\subsection{Statistical analysis}\label{statistical-analysis}

All statistical analysis, plotting, and mapping was undertaken using the
R Statistical Software (v4.4.0; R Core Team (2024)), through the RStudio
Integrated Development Environment v2023.12.1.402 (RStudio, 2024). All
plots and maps were created using the R package \texttt{ggplot2}
(v3.5.1; Wickham (2016)). Country and regional maps were created using
data from the \texttt{rnaturalearth} package (Massicotte and South,
2023) and other maps using ESRI imagery and the \texttt{OpenStreetMap}
package (Fellows, 2023). Generalized Linear Mixed Models (R package
\texttt{glmmTMB} v1.1.9; Brooks et al. (2017)) were used to investigate
the relation between the change in WEP (before - after treatment) and
treatment and riparian sampling location for each of the four sources of
WEP. Year and riparian area were included as crossed random factors to
control for the variability between years and riparian areas.
Additionally, when investigating the change in biomass WEP the WEP prior
to the treatment was included as a covariate because the magnitude of
the difference (i.e., before - after) is directly related to the amount
initially available.

The interaction term was removed if there were no significant
interactions between the main effects (p \textless0.5). When a main
effect or interaction were significant post-hoc pairwise comparisons
with a Benjamini-Hochberg p-value adjustment was used (\texttt{emmeans}
v1.10.1; Lenth (2024)). Model assumptions were assessed using DHARMa
residual plots (\texttt{DHARMa} v0.4.6; Hartig (2022)), main effects
were tested for collinearity (\texttt{performance} v0.12.2; Lüdecke et
al. (2021)), and results were presented as type III ANOVA (\texttt{car}
v3.1.2; Fox and Weisberg (2019)). For each unique source of WEP, the
null hypotheses are that there is no difference in the net WEP between
treatments or riparian sampling locations and there is no interaction
between these two factors.

\section{Results and Discussion}\label{results-and-discussion}

\subsection{Vertical and longitudinal profiles of
P}\label{vertical-and-longitudinal-profiles-of-p}

The four distinctive sources of P demonstrate is strong vertical
stratification in both the concentration and total WEP
(Figure~\ref{fig-vertical-wep}). The median concentrations in the
vegetation sources were 82.8 and 39.0 \(mg~kg^{-1}\) for the biomass and
litter components, respectively, which is more than an order of
magnitude greater than the soil components (0.9 and 3.4 \(mg~kg^{-1}\);
Ah and organic, respectively). Considerable variability in the WEP
concentration in the biomass and litter sources were observed with
interquartile ranges (IQR) of 54.3 and 32.9 \(mg~kg^{-1}\) for the
biomass and litter sources, respectively. In contrast, the IQR for the
organic and Ah sources was \textless2.5 \(mg~kg^{-1}\). Overall, in
terms of the total amount of WEP, the top 10 cm of the Ah horizon is the
largest source of WEP (42.5 \(mg~m^{-2}\)) followed by the biomass (26.3
\(mg~m^{-2}\)), organic layer (14.3 \(mg~m^{-2}\)), and lastly the
litter (13.7 \(mg~m^{-2}\)). The vertical profile of WEP in riparian
areas (Figure~\ref{fig-vertical-wep}) observed in this study supports
the concept that a soil test P alone is likely missing a large
proportion of the near-surface P that can be potentially lost during the
spring snowmelt (Liu et al., 2019a; b; Cober et al., 2019). The
substantial proportion of WEP above the soil surface provides evidence
that managing the biomass in riparian areas in autumn may reduce the
contribution of P lost directly from this area during spring.
Specifically, the harvesting of this biomass results in an export of P
which can maintain or enhance the buffering or storage capacity of P
derived from upslope sources further improving downstream water quality
(Kelly et al., 2007; Hille et al., 2019).

\begin{figure}[H]

\centering{

\pandocbounded{\includegraphics[keepaspectratio]{index_files/figure-latex/notebooks-04_Vertical_profile-fig-vertical-wep-output-2.png}}

}

\caption{\label{fig-vertical-wep}Vertical and longitudinal profiles of
a) WEP concentration and b) WEP content in the riparian areas prior to
grazing and mowing treatments.}

\end{figure}%

\textsubscript{Source:
\href{https://alex-koiter.github.io/riparian-grazing-manuscript/notebooks/04_Vertical_profile-preview.html\#cell-fig-vertical-WEP}{Vertical
profile of WEP}}

The median concentrations are similar between the upper, mid, and lower
positions in the biomass, organic, and Ah sources. An increase of
\textasciitilde20 \(mg~kg^{-1}\) in WEP is observed in the litter from
the upper to lower riparian sampling location. The total WEP does show
an impact of the location within the riparian area. For the biomass and
litter sources the lower riparian locations had greater amounts of WEP
whereas the organic and Ah sources had greater amount of WEP in the
upper riparian locations. The amount of variability is greatest in the
Ah (IQR = 32.0 \(mg~kg^{-1}\)) and biomass (IQR = 23.3 \(mg~kg^{-1}\))
sources. The variability of the other two sources were similar with IQRS
of 15.6 and 14.3 \(mg~kg^{-1}\) for the litter and organic layer,
respectively. The longitudinal gradient of WEP shows an inverted
symmetry where the biomass WEP is largest near the lower sampling
location and the Ah soil WEP is larger in the upper sampling location
adjacent to the fields (Figure~\ref{fig-vertical-wep} b). The high soil
water content in the lower location creates conditions that favor high
biomass production (\quartosuppfigref{suppfig-bd-plot}) coupled with
high biomass WEP concentrations (Figure~\ref{fig-vertical-wep} a)
resulting in a considerable source of P. The higher amount of WEP in the
Ah soil in the upper locations of the riparian area is due to the higher
bulk density (\quartosuppfigref{suppfig-bd-plot}) and higher WEP
concentration (Figure~\ref{fig-vertical-wep} a). The higher bulk density
is most likely due to the lower soil organic matter content and the
higher WEP concentration may be related to the interception of P-rich
runoff from upslope areas (Tomer et al., 2007). Understanding and
quantifying the sources and patterns of P within riparian areas is a key
part of assessing the risk of P loss and designing effective management
plans (Reid et al., 2018).

\subsection{Impacts of grazing and mowing on P
sources}\label{impacts-of-grazing-and-mowing-on-p-sources}

Results of the ANOVA show a significant effect of treatment on the net
biomass WEP (X\textsuperscript{2} = 24.8, df = 3, p \textless{} 0.001)
and riparian location (X\textsuperscript{2} = 15.7, df = 2, p
\textless{} 0.001). The net biomass WEP for the high-density grazing and
mowing treatments were similar (p\textgreater0.05) but significantly
(p\textless0.05) different from the control and graze treatments
(Figure~\ref{fig-vegetation-wep} a and Table~\ref{tbl-biomass-posthoc}).
The mowing and high density grazing reduced the average WEP amount by
7.4 and 4.2 \(mg~m^{-2}\) relative to the control, respectively. The
reduction in biomass WEP was significantly (p\textless0.05) greater in
the lower sampling locations as compared to the upper and mid locations
(Figure~\ref{fig-vegetation-wep} b and Table~\ref{tbl-biomass-posthoc})
with a difference in average WEP of 10.2 \(mg~m^{-2}\) between the lower
and upper locations of the riparian area.

\begin{figure}[H]

\centering{

\pandocbounded{\includegraphics[keepaspectratio]{index_files/figure-latex/notebooks-01_Biomass_analysis-fig-vegetation-wep-output-2.png}}

}

\caption{\label{fig-vegetation-wep}Change in riparian biomass WEP
following grazing or mowing in each riparian location. Within each plot
significant differences (p\textless0.05) between treatments or riparian
locations are denoted with different letters. Lower sampling locations
are adjacent to the edge of the waterbody and Upper locations are
adjacent to the field.}

\end{figure}%

\textsubscript{Source:
\href{https://alex-koiter.github.io/riparian-grazing-manuscript/notebooks/01_Biomass_analysis-preview.html\#cell-fig-vegetation-WEP}{Riparian
vegetation WEP in response to grazing}}

\begin{longtable}[]{@{}cccccc@{}}

\caption{\label{tbl-biomass-posthoc}Results of the post-hoc pairwise
comparisons with a Benjamini-Hochberg p value adjustment for differences
in the net biomass WEP (\(mg~m^{-2}\)) between the four treatments and
three riparian sampling locations.}

\tabularnewline

\toprule\noalign{}
Contrast & Estimate & SE & df & t ratio & p value \\
\midrule\noalign{}
\endhead
\bottomrule\noalign{}
\endlastfoot
\multicolumn{6}{@{}c@{}}{%
Treatment} \\
Control - High Graze & −4.83 & 2.42 & 132 & −2.00 & 0.072 \\
Control - Mow & −8.52 & 2.42 & 132 & −3.52 & 0.002 \\
Control - Regular Graze & 2.47 & 2.40 & 132 & 1.03 & 0.306 \\
High Graze - Mow & −3.69 & 2.43 & 132 & −1.51 & 0.159 \\
High Graze - Regular Graze & 7.30 & 2.42 & 132 & 3.02 & 0.006 \\
Mow - Regular Graze & 10.99 & 2.42 & 132 & 4.55 & \textless0.001 \\
\multicolumn{6}{@{}c@{}}{%
Location} \\
Lower - Middle & −7.94 & 2.43 & 132 & −3.26 & 0.002 \\
Lower - Upper & −9.82 & 2.57 & 132 & −3.83 & \textless0.001 \\
Middle - Upper & −1.87 & 2.11 & 132 & −0.89 & 0.377 \\

\end{longtable}

\textsubscript{Source:
\href{https://alex-koiter.github.io/riparian-grazing-manuscript/notebooks/01_Biomass_analysis-preview.html\#cell-tbl-biomass-posthoc}{Riparian
vegetation WEP in response to grazing}}

There were no significant impacts of either treatment
(X\textsuperscript{2} = 1.15, df = 3, p = 0.23) or riparian location
(X\textsuperscript{2} = 4.30, df = 2, p = 0.56) on the amount of litter
WEP (Figure~\ref{fig-litter-wep}). With respect to the WEP concentration
in the organic layer the ANOVA detected no significant difference
between riparian location (X\textsuperscript{2} = 0.57, df = 2, p =
0.75) but a significant (X\textsuperscript{2} = 8.24, df = 3, p = 0.04)
effect of treatment. However, the post-hoc pairwise comparisons
(Table~\ref{tbl-organic-posthoc}) found no significant differences (p
\textless0.05) between the three riparian positions. Lastly, there was
no significant effect of treatment (X\textsuperscript{2} = 2.59, df = 3,
p = 0.46) or riparian position (X\textsuperscript{2} = 1.17, df = 2, p =
0.56) in the concentration of WEP in the top 10 cm of the Ah horizon
(Figure~\ref{fig-soil-wep}). There was considerable variation across all
treatments and riparian locations in all four P sources. This high
variability in WEP amount/concentration is best reflected in the control
treatment where the expected difference is 0. Still, WEP losses and
gains were measured despite no treatment being applied.

\begin{figure}[H]

\centering{

\pandocbounded{\includegraphics[keepaspectratio]{index_files/figure-latex/notebooks-02_Litter_analysis-fig-litter-wep-output-2.png}}

}

\caption{\label{fig-litter-wep}Change in riparian litter WEP following
grazing or mowing in each of the riparian locations. No significant
effect of treatment or riparian location on the litter WEP content was
detected. Lower sampling locations are adjacent to the edge of the
waterbody and Upper locations are adjacent to the field.}

\end{figure}%

\textsubscript{Source:
\href{https://alex-koiter.github.io/riparian-grazing-manuscript/notebooks/02_Litter_analysis-preview.html\#cell-fig-litter-WEP}{Riparian
litter WEP in response to grazing}}

\begin{figure}[H]

\centering{

\pandocbounded{\includegraphics[keepaspectratio]{index_files/figure-latex/notebooks-03_Soils_analysis-fig-organic-wep-output-1.png}}

}

\caption{\label{fig-organic-wep}Change in riparian organic layer WEP
concentration following grazing or mowing in each of the riparian
locations. A significant effect of treatment was detected; however, the
post-hoc analysis was not able to detect any significant (p \textless{}
0.05) pairwise contrasts. Lower sampling locations are adjacent to the
edge of the waterbody and Upper locations are adjacent to the field.}

\end{figure}%

\textsubscript{Source:
\href{https://alex-koiter.github.io/riparian-grazing-manuscript/notebooks/03_Soils_analysis-preview.html\#cell-fig-organic-WEP}{Riparian
organic and mineral soil WEP in response to grazing}}

\begin{longtable}[]{@{}llllll@{}}

\caption{\label{tbl-organic-posthoc}Results of the post-hoc pairwise
comparisons with a Benjamini-Hochberg p value adjustment for differences
in the net organic layer WEP (\(mg~kg^{-1}\)) between the four
treatments.}

\tabularnewline

\toprule\noalign{}
Contrast & Estimate & SE & df & t ratio & p value \\
\midrule\noalign{}
\endhead
\bottomrule\noalign{}
\endlastfoot
Control - Graze & −1.49 & 0.59 & 135 & −2.50 & 0.066 \\
Control - High Graze & −0.63 & 0.59 & 135 & −1.05 & 0.353 \\
Control - Mow & −1.38 & 0.59 & 135 & −2.32 & 0.066 \\
Graze - High Graze & 0.86 & 0.59 & 135 & 1.45 & 0.299 \\
Graze - Mow & 0.11 & 0.59 & 135 & 0.18 & 0.856 \\
High Graze - Mow & −0.75 & 0.59 & 135 & −1.27 & 0.311 \\

\end{longtable}

\textsubscript{Source:
\href{https://alex-koiter.github.io/riparian-grazing-manuscript/notebooks/03_Soils_analysis-preview.html\#cell-tbl-organic-posthoc}{Riparian
organic and mineral soil WEP in response to grazing}}

\begin{figure}[H]

\centering{

\pandocbounded{\includegraphics[keepaspectratio]{index_files/figure-latex/notebooks-03_Soils_analysis-fig-soil-wep-output-1.png}}

}

\caption{\label{fig-soil-wep}Change in riparian Ah layer (0-10cm) WEP
concentration following grazing or mowing in each of the riparian
locations. No significant effect of treatment or location was detected.
Lower sampling locations are adjacent to the edge of the waterbody and
Upper locations are adjacent to the field.}

\end{figure}%

\textsubscript{Source:
\href{https://alex-koiter.github.io/riparian-grazing-manuscript/notebooks/03_Soils_analysis-preview.html\#cell-fig-soil-WEP}{Riparian
organic and mineral soil WEP in response to grazing}}

The results of this study suggest that short-term autumn high density
grazing may be a potential management tool that can reduce the amount of
P lost directly from the riparian area (Figure~\ref{fig-vegetation-wep}
a). In addition to managing P loss, grazing riparian areas can also
provide an essential source of forage, particularly during drought.
Mechanized harvesting of biomass will also achieve this reduction in P
loss (Figure~\ref{fig-vegetation-wep} a) if the landscape and soil
conditions are favorable. Despite the cycling of nutrients by the
removal of P through grazing of biomass
(Figure~\ref{fig-vegetation-wep}) and the deposition through excretion
no differences were detected in the litter and Ah sources of P
(Figure~\ref{fig-litter-wep}, and \ref{fig-soil-wep}). The ANOVA did
detect a significant effect of treatment on the organic layer WEP;
however, the pairwise comparisons were not able to detect any
significant differences and the exact nature of the impact of the
treatments remains unclear. The ability to detect changes in the WEP
sources in riparian areas is difficult due to spatial variability in
both the inherent and post-grazing treatment. Even within the control
plots, both net addition and removal of WEP were measured and in many
cases the amount of variability was similar across treatments. This
inherent variability (i.e., pre-grazing) is a result of a combination of
hydrological factors like ground water fluctuations, soil attributes
such as texture, ecological dynamics involving plant community
composition, and anthropogenic influences like historical land
management practices (McClain et al., 2003; Vidon et al., 2010). In
particular, the species cover information
(\quartosuppfigref{suppfig-plant-plot}) demonstrates a wide range in
species composition and abundance, this coupled with the variation in P
release with different vegetation species may explain some of the
observed variability (Cober et al., 2018).

\subsection{Sources of variability and uncertainty in P
sources}\label{sources-of-variability-and-uncertainty-in-p-sources}

The Prairie pothole wetlands regularly experience high water levels in
the early spring. Observations made adjacent to one plot between Oct
2020 and May 2021 showed that the lower, mid, and upper sampling points
would have experienced inundation for approximately 21, 11, and zero
days, respectively. The annual weather conditions and topography of
riparian areas surrounding the wetlands will have an large impact on the
length and extent of flooding. Prolonged contact with water has been
shown to increase the amount of WEP in both soil (Young and Briggs,
2008) and vegetation (Lozier and Macrae, 2017) and also may explain some
of the observed variability. As reported by Podolsky and Schindler
(1993), the soils surrounding these potholes are typically low in
CaCo\textsubscript{3} and have a neutral to slight alkaline pH. In this
pH range (\textasciitilde6.5 to 7.5) P availability is typically at its
highest and not expected to precipitate with Ca. A more detailed soil
chemical analysis, particularly Fe and Mn, along with soil saturation
duration information (i.e., redox) is needed to fully assess the
potential for P loss during the spring (Walton et al., 2020). The WEP
protocol used for both soil and vegetation samples are not likely to
capture mobilize redox-sensitive P from the soil (Walton et al., 2020)
or enhanced P leaching from vegetation (Lozier and Macrae, 2017).
Similarly, the WEP protocol also does not capture the enhanced P release
from soil and vegetation that results repeated freeze-thaw cycles (Liu
et al., 2013; Lozier and Macrae, 2017). Temperature sensors placed at
the soil surface adjacent to one plot recorded four freeze-thaw cycles
between Oct 2020 and May 2021, surface temperatures fluctuations are
moderated in this region by the continental climate and relatively
persistent snow pack. Both the prolonged contact with water and
freeze-thaw cycles are are not captured in the WEP protocols and likely
result in an underestimation of the potential for P loss from the each
of the four distinctive source of P in riparian areas.

Post-grazing treatment, the added urine and manure create additional
hotspots of P that may carry forward to subsequent years (Subedi et al.,
2020; Donohoe et al., 2021). The single 0.25 \(m^2\) sampling quadrate
within each riparian location may have been insufficient to capture the
spatial variability. Therefore, larger composite and/or several sampling
locations within each upper, middle and lower locations are recommended.
Appropriate sampling design becomes critical as the scale of observation
of similar research increases to the farm scale, and so will the amount
and source of variability. As the scope of research is expanded to the
farm level, the importance of using an appropriate sampling design
becomes increasingly critical (Hale et al., 2014).

\subsection{Managment implications}\label{managment-implications}

Autumn was selected for the mowing and grazing treatments for three
reasons. The first was to reduce the amount of biomass P available that
can contribute to the P loss during the spring snowmelt. Secondly, the
drier soil conditions reduce the amount of pugging and soil compaction,
which limits the disruption of soil structure and damage to plants
(Batey, 2009). Lastly, the prairie potholes and associated riparian
areas are important breeding habitats for migratory birds, and
late-season grazing may reduce the ecological impact (Stanley and Knopf,
2002). However, the type of grazing system (timing, stocking rate, and
density, etc.) may impact habitat quality and breeding success
(Carnochan et al., 2018; Hansen et al., 2019; Kraft et al., 2021).
Corridor fencing at the edge of the waterbody and alternative water
sources were used in this study to limit livestock access to prevent
bank erosion and protect water quality (e.g., direct deposition)
(Dauwalter et al., 2018). Scaling this up to the farm level might be
expensive (fencing infrastructure) (Aarons et al., 2013) and
time-consuming (short-term grazing), especially in prairie pothole
regions where there are numerous and small riparian areas (Manitoba
Agriculture, 2024). The long-term impacts of repeated grazing of
riparian areas also needs to be considered. From a nutrient loss
reduction perspective, a shift in the magnitude of P sources could be
expected as less biomass is available to be added to the litter source,
affecting the organic layer and Ah sources of P. The regular inclusion
of cattle will also introduce a new manure source of P, which can
spatially redistribute P and initially be more water soluble and readily
transported (Franzluebbers et al., 2019). Grazing can also reduce the
litter layer through trampling increasing the soil-vegetation contact,
and speeding up the decomposition process. These changes in biomass and
litter quantities may result in changes to habitat structure.
Understanding how forage management directly impacts both the plant and
soil P dynamics is important for understanding both the agronomic and
environmental P considerations (Subedi et al., 2020).

\section{Conclusion}\label{conclusion}

Biomass and litter are significant sources of near-surface WEP in
riparian areas that historically have been disregarded as necessary.
Management of the biomass prior to the onset of winter conditions in
cold climates has the potential to reduce the amount of P directly lost
during the spring snowmelt and maintain or enhance the nutrient
buffering capacity. The results from this experiment demonstrated that
short-term high-density cattle grazing and mowing both resulted in a
reduction in the amount of biomass WEP, particularly in the lower
riparian locations. The grazing and mowing treatments had no detectable
effect on the other three near-surface sources of WEP. However,
detecting changes in the near-surface sources of WEP is challenging due
to high spatial variability.

Comparatively less riparian research has occurred in landscapes that
experience a cold climate with strong temperature seasonality (e.g.,
Canadian prairies). In these regions, the runoff and nutrient losses
occur predominately during the spring snowmelt period. The repeated FTC
of the vegetation and soils increases the potential P losses during this
key time. Continued research to identify, quantify, and manage these
sources of P to improve water quality remains a priority. In addition
improving water quality, the development of riparian management
strategies should prioritize the protection other ecological goods and
services and recognize these areas as an integral part of the farm.

\section*{Acknowledgements}\label{acknowledgements}
\addcontentsline{toc}{section}{Acknowledgements}

This project was undertaken with the financial support of the Government
of Canada through the federal Department of Environment and Climate
Change and a Lake Winnipeg Basin Program grant awarded to the Manitoba
Association of Watersheds. Additional research funding was provided
through a Brandon University Research Committee grant awarded to AK.
Thank you to A. Avila, M. Luna, C Sobchuk, and A. Tan for all the help
with lab and field work. Special thanks to the Manitoba Beef and Forage
Initiatives research farm staff for the use of their facilities and
managing the cattle grazing and mowing treatments. Lastly, thank you to
R. Canart and M. Elsinger for helping to develop the experimental
design.

\section*{Data availability}\label{data-availability}
\addcontentsline{toc}{section}{Data availability}

Data and source code for analysis and manuscript available on GitHub:
\url{https://github.com/alex-koiter/riparian-grazing-manuscript}

\section*{Conflict of interest
statement}\label{conflict-of-interest-statement}
\addcontentsline{toc}{section}{Conflict of interest statement}

The authors have no competing interests to declare that are relevant to
the content of this article.

\section*{Author contributions}\label{author-contributions}
\addcontentsline{toc}{section}{Author contributions}

The authors confirm contribution to the paper as follows: study
conception and design: A. Koiter; data collection: T. Malone; analysis
and interpretation of results: A. Koiter; draft manuscript preparation:
A. Koiter and T. Malone. All authors reviewed the results and approved
the final version of the manuscript.

\section*{References}\label{references}
\addcontentsline{toc}{section}{References}

\phantomsection\label{refs}
\begin{CSLReferences}{1}{1}
\vspace{1em}

\bibitem[\citeproctext]{ref-aarons2013}
Aarons, S.R., A.R. Melland, and L. Dorling. 2013. Dairy farm impacts of
fencing riparian land: Pasture production and farm productivity. Journal
of Environmental Management 130: 255--266. doi:
\href{https://doi.org/10.1016/j.jenvman.2013.08.060}{10.1016/j.jenvman.2013.08.060}.

\bibitem[\citeproctext]{ref-batey2009}
Batey, Tom. 2009. Soil compaction and soil management {\textendash} a
review. Soil Use and Management 25(4): 335--345. doi:
\href{https://doi.org/10.1111/j.1475-2743.2009.00236.x}{10.1111/j.1475-2743.2009.00236.x}.

\bibitem[\citeproctext]{ref-beck2018}
Beck, H.E., N.E. Zimmermann, T.R. McVicar, N. Vergopolan, A. Berg, et
al. 2018. Present and future Köppen-Geiger climate classification maps
at 1-km resolution. Scientific Data 5(1): 180214. doi:
\href{https://doi.org/10.1038/sdata.2018.214}{10.1038/sdata.2018.214}.

\bibitem[\citeproctext]{ref-brooks2017}
Brooks, M.E., K. Kristensen, K.J. van Benthem, A. Magnusson, C.W. Berg,
et al. 2017. {glmmTMB} balances speed and flexibility among packages for
zero-inflated generalized linear mixed modeling. The R Journal 9(2):
378--400. doi:
\href{https://doi.org/10.32614/RJ-2017-066}{10.32614/RJ-2017-066}.

\bibitem[\citeproctext]{ref-carlyle2001}
Carlyle, G.C., and A.R. Hill. 2001. Groundwater phosphate dynamics in a
river riparian zone: Effects of hydrologic flowpaths, lithology and
redox chemistry. Journal of Hydrology 247(3): 151--168. doi:
\href{https://doi.org/10.1016/S0022-1694(01)00375-4}{10.1016/S0022-1694(01)00375-4}.

\bibitem[\citeproctext]{ref-carnochan2018}
Carnochan, S.J., C.C. De Ruyck, and N. Koper. 2018. Effects of
twice-over rotational grazing on songbird nesting success in years with
and without flooding. Rangeland Ecology \& Management 71(6): 776--782.
doi:
\href{https://doi.org/10.1016/j.rama.2018.04.013}{10.1016/j.rama.2018.04.013}.

\bibitem[\citeproctext]{ref-cober2018}
Cober, J.R., M.L. Macrae, and L.L. Van Eerd. 2018. Nutrient release from
living and terminated cover crops under variable freeze{\textendash}thaw
cycles. Agronomy Journal 110(3): 1036--1045. doi:
\href{https://doi.org/10.2134/agronj2017.08.0449}{10.2134/agronj2017.08.0449}.

\bibitem[\citeproctext]{ref-cober2019}
Cober, J.R., M.L. Macrae, and L.L. Van Eerd. 2019. Winter phosphorus
release from cover crops and linkages with runoff chemistry. Journal of
Environmental Quality 48(4): 907--914. doi:
\href{https://doi.org/10.2134/jeq2018.08.0307}{10.2134/jeq2018.08.0307}.

\bibitem[\citeproctext]{ref-dauwalter2018}
Dauwalter, D.C., K.A. Fesenmyer, S.W. Miller, and T. Porter. 2018.
Response of riparian vegetation, instream habitat, and aquatic biota to
riparian grazing exclosures. North American Journal of Fisheries
Management 38(5): 1187--1200. doi:
\href{https://doi.org/10.1002/nafm.10224}{10.1002/nafm.10224}.

\bibitem[\citeproctext]{ref-donohoe2021}
Donohoe, G., D. Flaten, F. Omonijo, and K. Ominski. 2021. Short-term
impacts of winter bale grazing beef cows on forage production and soil
nutrient status in the eastern canadian prairies. Canadian Journal of
Soil Science 101(4): 717--733. doi:
\href{https://doi.org/10.1139/cjss-2021-0028}{10.1139/cjss-2021-0028}.

\bibitem[\citeproctext]{ref-environmentandclimatechangecanada2024}
Environment, and Climate Change Canada. 2024. Canadian Climate Normals.
\url{https://climate.weather.gc.ca/climate_normals/index_e.html}.

\bibitem[\citeproctext]{ref-fellows2023}
Fellows, I. 2023.
\href{https://CRAN.R-project.org/package=OpenStreetMap}{OpenStreetMap:
Access to open street map raster images}.

\bibitem[\citeproctext]{ref-fitch2003}
Fitch, L., B. Adams, and K. O'Shaughnessy. 2003. Caring for the green
zone: Riparian areas and grazing management. 3rd ed. Cows; Fish Program,
Lethbridge, Alberta.

\bibitem[\citeproctext]{ref-fox2019}
Fox, J., and S. Weisberg. 2019.
\href{https://socialsciences.mcmaster.ca/jfox/Books/Companion/}{An {R}
companion to applied regression}. Third. Sage, Thousand Oaks {CA}.

\bibitem[\citeproctext]{ref-franzluebbers2019}
Franzluebbers, A.J., M.H. Poore, S.R. Freeman, and J.R. Rogers. 2019.
Soil-surface nutrient distributions in grazed pastures of North
Carolina. Journal of Soil and Water Conservation 74(6): 571--583. doi:
\href{https://doi.org/10.2489/jswc.74.6.571}{10.2489/jswc.74.6.571}.

\bibitem[\citeproctext]{ref-habibiandehkordi2019}
Habibiandehkordi, R., D.A. Lobb, P.N. Owens, and D.N. Flaten. 2019.
Effectiveness of vegetated buffer strips in controlling legacy
phosphorus exports from agricultural land. Journal of Environmental
Quality 48(2): 314--321. doi:
\href{https://doi.org/10.2134/jeq2018.04.0129}{10.2134/jeq2018.04.0129}.

\bibitem[\citeproctext]{ref-habibiandehkordi2017}
Habibiandehkordi, R., D.A. Lobb, S.C. Sheppard, D.N. Flaten, and P.N.
Owens. 2017. Uncertainties in vegetated buffer strip function in
controlling phosphorus export from agricultural land in the canadian
prairies. Environmental Science and Pollution Research 24(22):
18372--18382. doi:
\href{https://doi.org/10.1007/s11356-017-9406-6}{10.1007/s11356-017-9406-6}.

\bibitem[\citeproctext]{ref-hale2014}
Hale, R., P. Reich, T. Daniel, P.S. Lake, and T.R. Cavagnaro. 2014.
Scales that matter: Guiding effective monitoring of soil properties in
restored riparian zones. Geoderma 228-229: 173--181. doi:
\href{https://doi.org/10.1016/j.geoderma.2013.09.019}{10.1016/j.geoderma.2013.09.019}.

\bibitem[\citeproctext]{ref-hansen2019}
Hansen, B.D., H.S. Fraser, and C.S. Jones. 2019. Livestock grazing
effects on riparian bird breeding behaviour in agricultural landscapes.
Agriculture, Ecosystems \& Environment 270-271: 93--102. doi:
\href{https://doi.org/10.1016/j.agee.2018.10.016}{10.1016/j.agee.2018.10.016}.

\bibitem[\citeproctext]{ref-hartig2022}
Hartig, F. 2022.
\href{https://CRAN.R-project.org/package=DHARMa}{DHARMa: Residual
diagnostics for hierarchical (multi-level / mixed) regression models}.

\bibitem[\citeproctext]{ref-hille2019}
Hille, S., D. Graeber, B. Kronvang, G.H. Rubæk, N. Onnen, et al. 2019.
Management options to reduce phosphorus leaching from vegetated buffer
strips. Journal of Environmental Quality 48(2): 322--329. doi:
\href{https://doi.org/10.2134/jeq2018.01.0042}{10.2134/jeq2018.01.0042}.

\bibitem[\citeproctext]{ref-kelly2007}
Kelly, J.M., J.L. Kovar, R. Sokolowsky, and T.B. Moorman. 2007.
Phosphorus uptake during four years by different vegetative cover types
in a riparian buffer. Nutrient Cycling in Agroecosystems 78(3):
239--251. doi:
\href{https://doi.org/10.1007/s10705-007-9088-4}{10.1007/s10705-007-9088-4}.

\bibitem[\citeproctext]{ref-kieta2019}
Kieta, K.A., and P.N. Owens. 2019. Phosphorus release from shoots of
phleum pretense l. After repeated freeze-thaw cycles and harvests.
Ecological Engineering 127: 204--211. doi:
\href{https://doi.org/10.1016/j.ecoleng.2018.11.024}{10.1016/j.ecoleng.2018.11.024}.

\bibitem[\citeproctext]{ref-kieta2018}
Kieta, K.A., P.N. Owens, D.A. Lobb, J.A. Vanrobaeys, and D.N. Flaten.
2018. Phosphorus dynamics in vegetated buffer strips in cold climates: A
review. Environmental Reviews 26(3): 255--272. doi:
\href{https://doi.org/10.1139/er-2017-0077}{10.1139/er-2017-0077}.

\bibitem[\citeproctext]{ref-kraft2021}
Kraft, J.D., D.A. Haukos, M.R. Bain, M.B. Rice, S. Robinson, et al.
2021. Using grazing to manage herbaceous structure for a
heterogeneity-dependent bird. The Journal of Wildlife Management 85(2):
354--368. doi:
\href{https://doi.org/10.1002/jwmg.21984}{10.1002/jwmg.21984}.

\bibitem[\citeproctext]{ref-krall2023}
Krall, M., and P. Roni. 2023. Effects of livestock exclusion on stream
habitat and aquatic biota: A review and recommendations for
implementation and monitoring. North American Journal of Fisheries
Management 43(2): 476--504. doi:
\href{https://doi.org/10.1002/nafm.10863}{10.1002/nafm.10863}.

\bibitem[\citeproctext]{ref-lacas2005}
Lacas, J.-G., M. Voltz, V. Gouy, N. Carluer, and J.-J. Gril. 2005. Using
grassed strips to limit pesticide transfer to surface water: A review.
Agronomy for Sustainable Development 25(2): 253--266. doi:
\href{https://doi.org/10.1051/agro:2005001}{10.1051/agro:2005001}.

\bibitem[\citeproctext]{ref-lenth2024}
Lenth, R.V. 2024.
\href{https://CRAN.R-project.org/package=emmeans}{Emmeans: Estimated
marginal means, aka least-squares means}.

\bibitem[\citeproctext]{ref-liu2019}
Liu, J., H.M. Baulch, M.L. Macrae, H.F. Wilson, J.A. Elliott, et al.
2019a. Agricultural water quality in cold climates: processes, drivers,
management options, and research needs. Journal of Environmental Quality
48(4): 792--802. doi:
\href{https://doi.org/10.2134/jeq2019.05.0220}{10.2134/jeq2019.05.0220}.

\bibitem[\citeproctext]{ref-liu2013}
Liu, J., R. Khalaf, B. Ulén, and G. Bergkvist. 2013. Potential
phosphorus release from catch crop shoots and roots after
freezing-thawing. Plant and Soil 371(1): 543--557. doi:
\href{https://doi.org/10.1007/s11104-013-1716-y}{10.1007/s11104-013-1716-y}.

\bibitem[\citeproctext]{ref-liu2019a}
Liu, J., M.L. Macrae, J.A. Elliott, H.M. Baulch, H.F. Wilson, et al.
2019b. Impacts of cover crops and crop residues on phosphorus losses in
cold climates: a review. Journal of Environmental Quality 48(4):
850--868. doi:
\href{https://doi.org/10.2134/jeq2019.03.0119}{10.2134/jeq2019.03.0119}.

\bibitem[\citeproctext]{ref-lozier2017}
Lozier, T.M., and M.L. Macrae. 2017. Potential phosphorus mobilization
from above-soil winter vegetation assessed from laboratory water
extractions following freeze{\textendash}thaw cycles. Canadian Water
Resources Journal 42(3): 276--288. doi:
\href{https://doi.org/10.1080/07011784.2017.1331140}{10.1080/07011784.2017.1331140}.

\bibitem[\citeproctext]{ref-ludecke2021}
Lüdecke, D., M.S. Ben-Shachar, I. Patil, P. Waggoner, and D. Makowski.
2021. {performance}: An {R} package for assessment, comparison and
testing of statistical models. Journal of Open Source Software 6(60):
3139. doi:
\href{https://doi.org/10.21105/joss.03139}{10.21105/joss.03139}.

\bibitem[\citeproctext]{ref-lyu2021}
Lyu, C., X. Li, P. Yuan, Y. Song, H. Gao, et al. 2021. Nitrogen
retention effect of riparian zones in agricultural areas: A
meta-analysis. Journal of Cleaner Production 315: 128143. doi:
\href{https://doi.org/10.1016/j.jclepro.2021.128143}{10.1016/j.jclepro.2021.128143}.

\bibitem[\citeproctext]{ref-manitobaagriculture2023}
Manitoba Agriculture. 2023. Manitoba agriculture weather program.
\url{https://www.gov.mb.ca/agriculture/weather/manitoba-ag-weather.html}.

\bibitem[\citeproctext]{ref-manitobaagriculture2024}
Manitoba Agriculture. 2024. Agriculture rotational grazing.
\url{https://www.gov.mb.ca/agriculture/}.

\bibitem[\citeproctext]{ref-manitobabeefforageinitiatives2024}
Manitoba Beef \& Forage Initiatives. 2024. MBFI farm stations.
\url{https://www.mbfi.ca/farm-station}.

\bibitem[\citeproctext]{ref-massicotte2023}
Massicotte, P., and A. South. 2023.
\href{https://CRAN.R-project.org/package=rnaturalearth}{Rnaturalearth:
World map data from natural earth}.

\bibitem[\citeproctext]{ref-mcclain2003}
McClain, M.E., E.W. Boyer, C.L. Dent, S.E. Gergel, N.B. Grimm, et al.
2003. Biogeochemical hot spots and hot moments at the interface of
terrestrial and aquatic ecosystems. Ecosystems 6(4): 301312. doi:
\href{https://doi.org/10.1007/s10021-003-0161-9}{10.1007/s10021-003-0161-9}.

\bibitem[\citeproctext]{ref-mcguire2010}
McGuire, K.J., and J.J. McDonnell. 2010. Hydrological connectivity of
hillslopes and streams: Characteristic time scales and nonlinearities.
Water Resources Research 46(10): W10543. doi:
\href{https://doi.org/10.1029/2010WR009341}{10.1029/2010WR009341}.

\bibitem[\citeproctext]{ref-murphy1962}
Murphy, J., and J.P. Riley. 1962. A modified single solution method for
the determination of phosphate in natural waters. Analytica Chimica Acta
27: 31--36. doi:
\href{https://doi.org/10.1016/S0003-2670(00)88444-5}{10.1016/S0003-2670(00)88444-5}.

\bibitem[\citeproctext]{ref-nsengakumwimba2023}
Nsenga Kumwimba, M., J. Huang, M. Dzakpasu, K. De Silva, O.E. Ohore, et
al. 2023. An updated review of the efficacy of buffer zones in
warm/temperate and cold climates: Insights into processes and drivers of
nutrient retention. Journal of Environmental Management 336: 117646.
doi:
\href{https://doi.org/10.1016/j.jenvman.2023.117646}{10.1016/j.jenvman.2023.117646}.

\bibitem[\citeproctext]{ref-owens2007}
Owens, P.N., J.H. Duzant, L.K. Deeks, G.A. Wood, R.P.C. Morgan, et al.
2007. Evaluation of contrasting buffer features within an agricultural
landscape for reducing sediment and sediment-associated phosphorus
delivery to surface waters. Soil Use and Management 23(s1): 165--175.
doi:
\href{https://doi.org/10.1111/j.1475-2743.2007.00121.x}{10.1111/j.1475-2743.2007.00121.x}.

\bibitem[\citeproctext]{ref-podolsky1993}
Podolsky, G.P., and D. Schindler. 1993.
\href{https://www.manitoba.ca/agriculture/soil/soil-survey/pubs/fss02s00943.pdf}{Soils
of the manitoba zero tillage research association farm}. Winnipeg,
Manitoba.

\bibitem[\citeproctext]{ref-rcoreteam2024}
R Core Team. 2024. R: A language and environment for statistical
computing. \url{http://www.R-project.org}.

\bibitem[\citeproctext]{ref-reid2018}
Reid, K., K. Schneider, and B. McConkey. 2018. Components of phosphorus
loss from agricultural landscapes, and how to incorporate them into risk
assessment tools. Frontiers in Earth Science 6.
\url{https://www.frontiersin.org/article/10.3389/feart.2018.00135}.

\bibitem[\citeproctext]{ref-roberts2012}
Roberts, W.M., M.I. Stutter, and P.M. Haygarth. 2012. Phosphorus
retention and remobilization in vegetated buffer strips: A review.
Journal of Environmental Quality 41(2): 389--399. doi:
\href{https://doi.org/10.2134/jeq2010.0543}{10.2134/jeq2010.0543}.

\bibitem[\citeproctext]{ref-rstudio2024}
RStudio. 2024. RStudio: Integrated development environment for r.
\url{http://www.rstudio.org/}.

\bibitem[\citeproctext]{ref-schindler2012}
Schindler, D.W., R.E. Hecky, and G.K. McCullough. 2012. The rapid
eutrophication of lake winnipeg: Greening under global change. Journal
of Great Lakes Research 38, Supplement 3: 6--13. doi:
\href{https://doi.org/10.1016/j.jglr.2012.04.003}{10.1016/j.jglr.2012.04.003}.

\bibitem[\citeproctext]{ref-sharpley}
Sharpley, A.N., P.J.A. Kleinman, and J.L. Weld. 2006. Environmental soil
phosphorus indices. In: Carter, M.R. and Gregorich, E.G., editors. 2nd
ed. CRC Press, Boca Raton, FL, U.S.A

\bibitem[\citeproctext]{ref-stanley2002}
Stanley, T.R., and F.L. Knopf. 2002. Avian responses to late-season
grazing in a shrub-willow floodplain. Conservation Biology 16(1):
225--231. doi:
\href{https://doi.org/10.1046/j.1523-1739.2002.00269.x}{10.1046/j.1523-1739.2002.00269.x}.

\bibitem[\citeproctext]{ref-subedi2020}
Subedi, A., D. Franklin, M. Cabrera, A. McPherson, and S. Dahal. 2020.
Grazing systems to retain and redistribute soil phosphorus and to reduce
phosphorus losses in runoff. Soil Systems 4(4): 66. doi:
\href{https://doi.org/10.3390/soilsystems4040066}{10.3390/soilsystems4040066}.

\bibitem[\citeproctext]{ref-tomer2007}
Tomer, M.D., T.B. Moorman, J.L. Kovar, D.E. James, and M.R. Burkart.
2007. Spatial patterns of sediment and phosphorus in a riparian buffer
in western iowa. Journal of Soil and Water Conservation 62(5): 329--338.

\bibitem[\citeproctext]{ref-vidon2010}
Vidon, P., C. Allan, D. Burns, T.P. Duval, N. Gurwick, et al. 2010. Hot
spots and hot moments in riparian zones: Potential for improved water
quality management. Journal of the American Water Resources Association
46(2): 278--298. doi:
\href{https://doi.org/10.1111/j.1752-1688.2010.00420.x}{10.1111/j.1752-1688.2010.00420.x}.

\bibitem[\citeproctext]{ref-walton2020}
Walton, C.R., D. Zak, J. Audet, R.J. Petersen, J. Lange, et al. 2020.
Wetland buffer zones for nitrogen and phosphorus retention: Impacts of
soil type, hydrology and vegetation. Science of The Total Environment
727: 138709. doi:
\href{https://doi.org/10.1016/j.scitotenv.2020.138709}{10.1016/j.scitotenv.2020.138709}.

\bibitem[\citeproctext]{ref-wickham2016}
Wickham, H. 2016. ggplot2: Elegant graphics for data analysis.
Springer-Verlag, New York NY U.S.A.

\bibitem[\citeproctext]{ref-young2008}
Young, E.O., and R.D. Briggs. 2008. Phosphorus concentrations in soil
and subsurface water: A field study among cropland and riparian buffers.
Journal of Environmental Quality 37(1): 69--78. doi:
\href{https://doi.org/10.2134/jeq2006.0422}{10.2134/jeq2006.0422}.

\bibitem[\citeproctext]{ref-yu2019}
Yu, C., P. Duan, Z. Yu, and B. Gao. 2019. Experimental and model
investigations of vegetative filter strips for contaminant removal: A
review. Ecological Engineering 126: 25--36. doi:
\href{https://doi.org/10.1016/j.ecoleng.2018.10.020}{10.1016/j.ecoleng.2018.10.020}.

\end{CSLReferences}

\section*{Supplemental materials}\label{supplemental-materials}
\addcontentsline{toc}{section}{Supplemental materials}

\begin{suppfig}

\centering{

\pandocbounded{\includegraphics[keepaspectratio]{Workflow.png}}

}

\caption{\label{suppfig-workflow-plot}Workflow diagram showing the
experimental setup (yellow), field work (green), sample preparation
(brown), and laboratory analysis (blue).}

\end{suppfig}%

\begin{suppfig}

\centering{

\pandocbounded{\includegraphics[keepaspectratio]{plant_composition.png}}

}

\caption{\label{suppfig-plant-plot}Initial year (2019) cover assessment
using the foliar cover method for each plot within the four riparian
locations}

\end{suppfig}%

\begin{suppfig}

\centering{

\pandocbounded{\includegraphics[keepaspectratio]{supp-weather-plot-1.png}}

}

\caption{\label{suppfig-weather-plot}Average daily air temperature and
cumulative rainfall over the growing season over the three year study.
Red bars indicate sampling dates}

\end{suppfig}%

\begin{suppfig}

\centering{

\pandocbounded{\includegraphics[keepaspectratio]{supp-weights-bd-1.png}}

}

\caption{\label{suppfig-bd-plot}a) Mass of biomass and litter before
grazing and mowing (2019-2021) and b) the bulk density of the organic
layer and 10 cm Ah horizon (2023)}

\end{suppfig}%

\begin{suppfig}

\centering{

\pandocbounded{\includegraphics[keepaspectratio]{pair_plot.png}}

}

\caption{\label{suppfig-pairs-plot}Generalized pairs plot showing the
data and relationships between WEP concentration between the different
sources of Phosphorus at the lower (purple), middle (blue), and lower
(green) topographic positions. Data set only includes samples collected
before grazing and mowing treatments were applied. Corr indicates the
pearson correlation coefficient. *** p-value \textless{} 0.001, **
p-value \textless{} 0.01, ** p-value \textless{} 0.05.}

\end{suppfig}%

\begin{suppfig}

\centering{

\pandocbounded{\includegraphics[keepaspectratio]{P_conc_year.png}}

}

\caption{\label{suppfig-pairs-plot}Mean and standard deviation WEP
concentration for each of the different sources of Phosphorus at each
topographic position over the three year period of observations. Data
set only includes samples collected before grazing and mowing treatments
were applied.}

\end{suppfig}%




\end{document}
